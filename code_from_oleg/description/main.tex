%%%%%%%%%%%%%%%%%%%%%%%%%%%%%%%%%%%%%%

\documentclass[12pt]{article}
%\documentstyle[12pt,psfig,epsf]{article}

\hoffset=-15mm \voffset=-25mm \textwidth=165mm \textheight=245mm
\usepackage{graphicx}
\usepackage{amsmath}
\usepackage{amssymb}
%\usepackage{bigcircle}
\usepackage{wrapfig}
\usepackage{indentfirst}
\usepackage{color}
\usepackage{subfigure}

\renewcommand{\Re}{\operatorname{Re}}
\renewcommand{\Im}{\operatorname{Im}}

\begin{document}

\vskip 0.5cm \centerline{\bf\Large What's inside Oleg's code}
\vskip 1cm

\section{Double-differential cross-section of SDD as a function of $M_X^2$ for fixed $t$}
\begin{figure}[!h]
\centering
\includegraphics[width=.8\textwidth]{figures/SDD.eps}
\caption{Double-differential SDD cross-section.}
\label{fig:diagrams}
\end{figure}
This is similar to what is shown in Fig.8 of~\cite{Jenkovszky11}.
The cross-section implemented in the code is very similar to eq.(21) of~\cite{Jenkovszky11} and reads:
$$
\frac{d^2\sigma}{dt dM_X^2}=A_0 \frac{1}{(1. - t/0.71)^4} \left(\frac{s}{M_X^2}\right)^{2(1.08 + 0.25t) - 2} \sigma_{PpN}(M_X^2,t) KF(x,t,M_X^2)
$$

$$
A_0=977.441
$$

$$
\sigma_{PpN}(M_X^2,t)=\frac{\Im A(M_X^2,t)}{m_\textrm{proton}}
$$

$$
\Im A(M_X^2,t) = \sum_{n=1}^{3}\frac{1}{(1. - t/0.71)^{4(n+1)}} \frac{\Im\alpha(M_X^2)}{\left[2 n +0.5-\Re\alpha(M_X^2)\right]^2+\left[\Im\alpha(M_X^2)\right]^2}
$$

$$
\Im\alpha(s) = s^\delta \sum_{j=0}^{2} c_i (1-s_i/s)^{\Re\alpha(s_i)} \theta(s-s_i) 
$$

\begin{equation*}
\begin{split}
\Re\alpha(s) =  \alpha_0 + 
\frac{s}{\pi}
\sum_{j=0}^{2} c_i
  &\left[\theta(s_i-s)\frac{\Gamma(1-\delta) \Gamma(1+\lambda_i)}{\Gamma(\lambda_i-\delta+2) s_i^{1-\delta}} C_i  + \right. \\ 
  &\left.\theta(s-s_i) \left(\pi s^{\delta-1} \left(\frac{s-s_i}{s}\right)^{\lambda_i} \frac{1}{\tan{\left[\pi (1-\delta)\right]}} - \frac{\Gamma(-\delta) \Gamma(1+\lambda_i)}{\Gamma(\lambda_i-\delta+1)} \frac{s_i^\delta}{s} D_i\right)  \right]
\end{split}
\end{equation*}

$$
C_i=\begin{cases}
_2F_1(1; 1-\delta; \lambda_i-\delta +2; s/s_i),&\text{if $\lambda_i-\delta +2 < 10$;}\\
_2F_1(1; 1-\delta; 9.999; s/s_i),&\text{else.}\\
\end{cases}
$$

$$
D^2_i=\begin{cases}
_2F_1(\delta-\lambda_i; 1; \delta+1;  s_i/s),&\text{if $\lambda_i-\delta < 10$;}\\
_2F_1(-9.999; 1; \delta+1; s_i/s),&\text{else.}\\
\end{cases}
$$

Here the $_2F_1$ is the Gauss' hypergeometric function.

$$
\alpha_0=-0.410839
$$

$$
c_0=  0.508309,
c_1 = 4.01083,
c_2 = 4558.44.
$$
  
$$
s_0=  1.151,
s_1 = 2.44,
s_2 = 11.7.
$$

$$
\delta=-0.460021
$$

$$
\lambda_0=  0.840083,
\lambda_1 = 2.09707,
\lambda_2 = 11.1778.
$$

Kinematic factor:
$$
KF(x,t,M_X^2) = \frac{x (1-x)^2}{\left(1 - 4m_\textrm{proton}^2x^2/t\right)^{3/2}(M_X^2-m_\textrm{proton}^2)};
$$

$$
x = \frac{-t}{M_X^2 - m_\textrm{proton}^2 - t};
$$


%   $$
%    A(s,t,Q^2,{M_v}^2)= \widetilde{A_s}e^{-i\frac{\pi}{2}\alpha_s(t)}\left(\frac{s}{s_{0}}\right)^{\alpha_s(t)}
%     e^{b_st - n_s\ln{\left(1+\frac{\widetilde{Q^2}}{\widetilde{Q_s^2}}\right)}}
%   $$
%   \begin{equation}
%   +\widetilde{A_h}e^{-i\frac{\pi}{2}\alpha_h(t)}\left(\frac{s}{s_{0}}\right)^{\alpha_h(t)}
%     e^{b_ht - (n_h+1)\ln{\left(1+\frac{\widetilde{Q^2}}{\widetilde{Q_h^2}}\right)}
%     +\ln{\left(\frac{\widetilde{Q^2}}{\widetilde{Q_h^2}}\right)} },
%     \label{eq:Amplitude_FFJS}
%     \end{equation}


\begin{thebibliography}{99}
\bibitem{Jenkovszky11} 1011.0664
\bibitem{Jenkovszky12} 1211.5841
\end{thebibliography}

\end{document}
